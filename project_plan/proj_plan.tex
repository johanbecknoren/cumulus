\documentclass[12pt]{article}
\usepackage{graphicx,url,placeins,algorithm,algorithmic,lipsum,amssymb,amsmath,subfigure,float}
\usepackage[hidelinks]{hyperref}
\usepackage[titletoc]{appendix}
\usepackage{pdfpages}
\usepackage{listings}
\usepackage{layout}
\usepackage[font=small,labelfont=bf]{caption}
\usepackage[utf8]{inputenc}
\graphicspath{{figures/}}
\setlength{\voffset}{0in}
\setlength{\headsep}{5pt}
\addtolength{\textheight}{2cm}
%\setlength{\footskip}{5pt}
%
\begin{document}

\title{\vskip -5em TNCG13\\ Project Plan - A plugin for creating and viewing clouds in the Maya viewport}   % type title between braces

\author{
 Johan Beck-Nor\'{e}n, johbe559@student.liu.se
 }
        \date{\today}    % type date between braces
        \maketitle
\section{Description}
We want to write a plug-in for Maya where you can create a cloud volume either procedurallty from interactive settings, or from loading in a mesh
and converting it to volume voxel scalar values.The volume can be rendered in realtime in the viewport by using colorboxes for ray directions, and ray
marching for accumulating densitys. A transfer function can be used to transform the density values to intensities matching that of clouds. A front-marching
algorithm should be used so that we can use early termination if we reach an intensity maximum before traversing the entire volume.
\end{document}
